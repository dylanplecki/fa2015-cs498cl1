\documentclass[10pt, leqno]{exam}
\usepackage{xspace}
\usepackage{amsmath}
\usepackage{enumerate}
\usepackage{verbatim}
\usepackage{float}
\usepackage{mathtools}
\renewcommand{\labelitemi}{--}
\renewcommand{\labelitemii}{$\bullet$}

\newcommand{\name}{Henry Lin - halin2}
\newcommand{\assignment}{Discussion chapter 7}
\renewcommand{\date}{October 14th}
\newcommand{\class}{CS 498cl1, Fall 2015}

%\setlength{\parskip}{\baselineskip}%

%\newcommand{\frownie}{=D}
%\newcommand{\smiley}{=(}
\pagestyle{headandfoot}
\firstpageheadrule
\runningheadrule
\firstpageheader{\textit{\class}}
                {\textit{\name}}
                {\textit{\date}}
\runningheader	{\textit{\class}}
                {\textit{\assignment}}
                {\textit{Page \thepage\ of \numpages}}
\firstpagefooter{}{}{}
\runningfooter{}{}{}

% Fonts and typesetting
\RequirePackage[charter]{mathdesign}
\usepackage{microtype}
\setlength\parindent{0pt}

\begin{document}

    \textbf{{\LARGE \assignment}}
%%%%%%%%%%%%%%%%%%%%%%%%%%%%%%%%%%%%%%%%%%%%%%%%%%%%%%%%%%%%%%%%%%%%%%%%%%%%%%%%

    \section*{Magic the Gathering, DAF}

    The rules are as follows. (Please see page 208.)
    \begin{itemize}
    \item The game is played with a deck of 60 cards.
    \item Cards can be in one of 4 places.
        \begin{itemize}
        \item The library
        \item The hand
        \item The table
        \item The graveyard (only used spell cards are placed here.
        See \ref{game:clean}.)
        \end{itemize}

    \item There are two kinds of cards: Spells and Lands.
        \begin{itemize}
        \item Lands have cost 0. At every turn, you could place \textit{at most}
        one land on the table. (See \ref{game:put}.) When they're placed on the table, they will
        remain there for the rest of the game.
        \item Spells have cost $k$ (from 1 to 6). In order for you to use
        a spell of cost $k$, you must have at least $k$ "untapped" lands on
        the table. Using a spell card "taps" $k$ lands on the table. They
        remain tapped for the rest of the turn. If you have enough untapped
        lands, you could play more spell cards. (See step \ref{game:play}.)

        At the end of the turn, the spell card is placed in the graveyard,
        and the "tapped" cards are "untapped".\footnote{In his book, David
        untaps the cards at the beginning of the turn.} (See step \ref{game:clean}.)
        \end{itemize}
    \end{itemize}

    Here are the step-by-steps of how this game is played. Assume we play
    this game for $T$ turns.
    \newcommand{\landsontable}{\textsc{LandsOnTable}\xspace}
    \begin{enumerate}[1.]
    \item Initialize the game with a hand of 7 cards, and 53 cards in the
    library. The number of each spell card and land card may vary. (See
    the practice problems.)
    \item \landsontable = 0
    \item For $T$ turns:
        \begin{enumerate}
        \item \textbf{Draw}. Draw a card from your library, and put it into your hand.
        \item \textbf{Put}. \label{game:put}
        If you have a land in your hand, you may put it on the table.
        \landsontable += 1.

        \item \textbf{Play}. \label{game:play}
        You can now play your spell cards. You can play
        as many spells $s_1, s_2, \dots, s_k$ as you want, provided that
        $$cost(s_1) + cost(s_2) \dots + cost(s_k) \leq \landsontable.$$
        Using spell $s_i$ taps $cost(s_i)$ lands on the table.

        \textit{Note that in problem 7.5 and 7.6, we use \textbf{at most} one spell
        per turn. Problem 7.5 and 7.6 vary between picking the cheapest
        spell card, and picking the most expensive spell card.}

        \item \textbf{Clean up}. \label{game:clean}
        Put each spell card played in the graveyard.
        Untap all of the lands played.
        \end{enumerate}
    \end{enumerate}
    Note that we're \textbf{never removing} lands from the table. Therefore,
    the number of lands on the table only increases as the game goes on.
    Furthermore, there is no advantage of having lands in your hand; it's only
    natural that we want to place a land on the table when we have the opportunity.
    (See step \ref{game:put}.)

\end{document}
